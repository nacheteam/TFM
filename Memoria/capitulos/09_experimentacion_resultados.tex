\chapter{Experimentación y Resultados}
\label{chapter:experimentacion-resultados}

En todas las secciones anteriores hemos descrito el funcionamiento de nuestro sistema de detección de anomalías y alarmas. Como hemos podido comprobar, este sistema es como un puzle en el que tenemos tres piezas: el algoritmo de obtención de puntuaciones, el algoritmo de etiquetado de anomalías y el algoritmo de detección de alarmas. En este puzle podemos cambiar las piezas para ver si funciona mejor o peor y encontrar la mejor alternativa. En la experimentación desarrollada se ha cambiado el algoritmo de obtención de puntuaciones, probando los modelos desarrollados en las secciones anteriores y algunos modelos clásicos que han sido aplicados día por día a los datos de test. Los algoritmos clásicos empleados han sido: 
\begin{itemize}
	\item Clustering-Based Local Outlier Factor (CBLOF)~\cite{he2003discovering}: este algoritmo basa su comportamiento en la agrupación dada por un algoritmo de clustering. Supongamos que tenemos el espacio dividido por clusters y que tenemos unos parámetros que nos definen cuándo un cluster es pequeño o grande. Entonces podemos tomar el cluster al que pertenece un punto, estudiar si es grande o pequeño y ver si los clusters que tiene alrededor son grandes o pequeños a su vez. Teniendo esto en cuenta si un punto pertenece a un cluster pequeño en tamaño y su cluster grande más cercano está a mucha distancia podemos deducir que este punto es una anomalía. Por tanto la puntuación de cada uno de los puntos irá relacionada con el tamaño del cluster al que pertenece y la distancia al cluster grande más cercano.
	\item Histogram-Based Outlier Detection (HBOS)~\cite{hbos}: este método supone que las variables o features son independientes entre sí, por lo que realiza para cada una de ellas un histograma y estudia si el valor de cada una de las variables para una instancia es o no frecuente. De los histogramas calculados puede derivarse una probabilidad de que se de cada uno de los valores para cada una de las variables. Posteriormente estas probabilidades se agregan haciendo el mínimo de todas ellas.
	\item Isolation Forest~\cite{liu2008isolation}: este algoritmo emplea un ensemble de árboles de decisión para otorgar una puntuación de anomalía a cada instancia. Pensemos en un sólo árbol de decisión y que estamos evaluando la instancia i-ésima. Vamos a tomar un feature aleatorio y vamos a intentar separar la instancia i-ésima del resto haciendo divisiones aleatorias en el rango de ese feature. Por ejemplo si el rango de nuestra variable fuese el intervalo $[0,1]$ una división aleatoria podría ser la dada por los intervalos $[0,0.24]$, $(0.24,1]$. Estudiamos en cuál de los intervalos cae nuestra instancia y continuamos dividiendo hasta que nos quedemos con la instancia i-ésima en un intervalo en el que sólo esté ella misma. Si una instancia es anómala conseguiremos aislarla del resto en mucho menos recorrido, por lo que la profundidad del árbol será menor. Por contra si la instancia no es anómala esto nos llevará seguramente más divisiones que en el caso anómalo. Ahora este comportamiento lo replicamos para muchos árboles y finalmente para cada instancia hacemos la media de la profundidad alcanzada por cada árbol.
	\item K-Nearest Neighbors (KNN)~\cite{10.1145/335191.335437}: este detector basa su comportamiento en la idea de encontrar los K vecinos más cercanos de una instancia y estudiar si son próximos o no a ella. De cada uno de los vecinos de una instancia podemos obtener la distancia de dicho vecino al dato que estamos evaluando. Teniendo esas K distancias existen varios agregadores que podemos emplear como por ejemplo la máxima distancia, la media o la mediana. En este caso se ha empleado como agregador la media.
	\item Lightweight on-line Detector of Anomalies (LODA)~\cite{10.1007/s10994-015-5521-0}: este algoritmo emplea una idea similar a HBOS, pues se apoya en el uso de histogramas para la detección de anomalías. LODA, al contrario que HBOS, no hace los histogramas de todas las variables, sino que hace proyecciones de una dimensión haciendo combinaciones lineales de las variables de forma aleatoria y ponderada. Es decir, si tenemos $D$ variables, vamos a tomar un vector de tamaño $D$ con números entre $0$ y $1$ (algunos serán cero) y vamos a multiplicar cada columna por su número correspondiente y sumamos todos los valores. De esta forma estamos convirtiendo $D$ valores de cada instancia en uno sólo. Sobre este valor vamos a realizar un histograma con el que podremos obtener para cada instancia una probabilidad. Repitiendo este comportamiento aleatorio $k$ veces obtendremos $k$ probabilidades para cada punto. Finalmente estas probabilidades se agregan mediante una media logarítmica de las k probabilidades, es decir, hacemos la media de los logaritmos de las probabilidades y le cambiamos el signo pues al ser logaritmos de números entre $0$ y $1$ estos valores serán negativos.
	\item Principal Component Analysis (PCA)~\cite{Shyu03anovel}: este método va a tomar el principio básico de PCA y vamos a aplicarlo en la detección de anomalías. PCA toma la descomposición en valores singulares de los datos para llevarlos a un espacio de menor dimensión que explique la mayor varianza posible. Este espacio se encuentra estudiando los vectores propios resultantes con mayores valores propios. Si hacemos esto vamos a obtener un hiperplano definido por los vectores propios. Si un dato es normal se va a adaptar a este hiperplano y por tanto caerá cercano a él en el espacio, mientras que una anomalía no se ajustará a este espacio y por tanto caerá lejos el mismo. Hallando la distancia de la instancia original al hiperplano encontrado por PCA podemos dar una puntuación de anomalía.
	\item Local Outlier Factor (LOF)~\cite{breunig2000lof}: este algoritmo intenta analizar la densidad de puntos que tiene una instancia con respecto a su vecindario. Es claro que si una instancia tiene una densidad baja de puntos con respecto a sus vecinos es porque ella misma está aislada con respecto a ellos y por tanto es posible que sea una anomalía.
	\item Feature Bagging LOF~\cite{10.1145/1081870.1081891}: Feature Bagging no es más que una técnica que pretende hacer un estimador más robusto que su versión simple. La idea es que se toma un modelo como base (en este caso LOF) y se entrena con proyecciones de menos variables que los datos originales produciendo al final una serie de scores para la misma instancia del conjunto de datos. Tras esto se hace la media de todas ellas para obtener un score final. Esta metodología no es única para LOF, sino que se puede emplear con cualquier estimador base. En este caso lo empleamos con LOF pues era el modelo simple (junto con Isolation Forest y LODA) que mejor resultado parecía obtener.
\end{itemize}