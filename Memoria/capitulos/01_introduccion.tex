\chapter{Introducción}
\label{chapter:introduccion}
\markboth{Introducción}{}
\pagenumbering{arabic}
\setcounter{page}{1}

Antes de comenzar con el desarrollo en sí del estudio acometido en este trabajo, vamos a contextualizar el mismo y vamos a establecer un marco de trabajo teórico previo a la experimentación, que nos otorgará de rigurosidad para la parte práctica del mismo.

El estudio realizado en este trabajo versa sobre la aplicación de estructuras de Aprendizaje Profundo (Deep Learning) para obtención y detección de anomalías, en concreto, en series temporales. Dentro de este trabajo se van a desarrollar las técnicas conocidas como Autoencoders y Redes Neuronales para predicción de series temporales.

Lo primero que atacaremos en este estudio es la definición de anomalía, para luego pasar a una introducción teórica de Estadística Multivariante y Machine Learning en general. Estas dos secciones nos van a aportar el rigor que necesitamos para adentrarnos teóricamente dentro del Deep Learning y entender los fundamentos de las arquitecturas de redes neuronales que aplicaremos en la práctica.

Tras esto se realizará una descripción de la experimentación realizada, los datos que se emplearán en dicha experimentación y los resultados de la misma. 

%%%%%%%%%%%%%%%%%%%%%%%%%%%%%%%%%%%%%%%%%%%%%%%%%

\section{Contextualización}

Lo primero que debemos de hacer antes de empezar, es establecer el problema u objetivo a resolver de este estudio. Para ello vamos a hacer una breve introducción a los datos (o al problema propuesto que es lo mismo en este ámbito) y a explicar por qué precisamos de un trabajo arduo y prolongado, es decir, por qué no es un problema trivial.

\subsection{Definición del problema}

El ámbito de trabajo va a ser el de las series temporales, pues el conjunto de datos que nos define el problema es una serie temporal. Esta serie temporal mide la sensórica de una máquina de la empresa ArcelorMittal, que no podemos especificar por motivos de privacidad. En este sentido tenemos 106 variables de tipo numérico con las que vamos a trabajar y 468 días de datos con una granularidad de una medida por segundo. Esto hace que el volumen de datos del que disponemos sea inmenso, haciendo que tenga sentido el uso del Deep Learning por la enorme cantidad de datos de entrenamiento de los que vamos a disponer.

Como hemos comentado los datos son medidas de sensores de una cierta máquina. Esta máquina experimenta errores graves de vez en cuando, que hacen que se deba detener completamente para labores de mantenimiento. Nuestro objetivo es ser capaces de detectar estas labores de mantenimiento mediante técnicas de detección de anomalías. El principio subyacente es sencillo: esperamos un comportamiento normal de la máquina en la mayoría del tiempo salvo cuando haya necesidad de un mantenimiento, momento en el cual la sensórica arrojará medidas alteradas que nos den pie a pensar en un posible fallo.

Este tipo de problemas son conocidos como mantenimiento predictivo, pues lo que pretenden precisamente es anticipar la necesidad de dichas labores.

Con esto dicho nuestro objetivo será tomar los datos de entrada (la sensórica) para nuestros modelos Deep Learning y, de alguna manera, saber diferenciar lo que son datos normales y datos anómalos.

\section{Contenido básico y fuentes}

% REVISAR CUANDO TERMINE EL CONTENIDO %

El trabajo contiene una primera sección en la que se incluye una introducción de Aprendizaje Automático orientado específicamente a nuestro problema. Para ello primero se hace una contextualización del concepto de aprendizaje así como los principios inductivos que guían el mismo hacia un buen resultado como por ejemplo el ERM o minimización del error empírico. Se aportan también algunas reflexiones y conceptos en cuanto a la aproximación de funciones, que no es más que el objetivo del aprendizaje automático. 

Todos estos conocimientos están basados en la teoría estadística de Vapnik y Chervonenkis que es brevemente repasada y en la que se dan cotas sobre el aprendizaje y su rendimiento. Esta introducción ha sido escrita basándose en tres libros: Learning from Data de Yaser Abu-Mostafa \cite{yaser_learning_2012}, Learning from Data de Cherkassky y Mulier \cite{cherkassky_learning_2007}  y Outlier Ensembles de Aggarwal y Sathe \cite{aggarwal_outlier_2017-1}.

Este marco nos dirige hacia la primera definición del concepto de anomalía que está basada en distancias y rangos intercuartil que se describen en el libro Outlier Analysis \cite{aggarwal_outlier_2017}. Además se introduce el tipo de datos que vamos a encontrar en el marco del estudio: las series temporales.

Para dar una definición alternativa y una buena base teórica para los modelos debemos hacer una breve introducción estadística. En esta introducción se define un vector aleatorio así como su función de densidad, su función característica y su función de distribución. Se refieren los conceptos de independencia y probabilidad y esperanza condicionada. Por último y aprovechando este contexto se enuncian y demuestran algunas desigualdades y fórmulas famosas. Este contenido viene dado por los apuntes de la asignatura Estadística Multivariante del grado en Matemáticas, los apuntes de la asignatura Procesos Estocásticos del grado en Matemáticas y el libro Probability Theory de M. Loève \cite{m_loeve_probability_1977}.

Tras esto puede ser introducido el concepto probabilístico y basado en densidad de una anomalía. Este concepto viene apoyado en el paper \cite{fabian_keller_hics_2012} que describe el algoritmo HICS.

Tras este marco teórico de aprendizaje, estadística multivariante y definiciones de anomalías introducimos los conocimientos necesarios de Deep Learning para el desarrollo y entendimiento de los modelos no supervisados implementados. Esta sección hace una revisión de cómo funcionan las redes neuronales, distintos tipos de capas empleados en las arquitecturas implementadas y algunas arquitecturas con mayor fundamentación académica como los Autoencoder. Esta sección se apoya en el libro Deep Learning de Goodfellow, Bengio y Courville \cite{goodfellow_deep_2016}, el libro Deep Learning with Tensorflow de Zaccone, Karim y Menshawy \cite{giancarlo_deep_2017} y la revisión sobre modelos Autoencoder de D. Charte, F. Charte, García, Mª J. del Jesús y F. Herrera \cite{david_practical_2018}.

La siguiente sección analiza y explica los modelos implementados con su código propio. Esta sección se apoya en varios artículos como \cite{lih_oh_automated_2018} o \cite{david_practical_2018} para obtener las arquitecturas empleadas y aplicarlas al ámbito no supervisado que necesitamos.

La siguiente sección explica cómo se integran los algoritmos de detección de anomalías para poder ser empleados para detectar los mantenimientos y cómo obtenemos etiquetas de los datos.

Tras esto se exponen los resultados obtenidos así como la metodología de la experimentación para finalizar con las conclusiones y trabajo futuro derivados del estudio.

\section{Objetivos}

Por todo lo descrito anteriormente el trabajo tiene los siguientes objetivos claros:

\begin{itemize}
	\item Desarrollar un marco teórico sobre el Machine Learning.
	\item Desarrollar un marco teórico sobre el Deep Learning.
	\item Estudiar el estado del arte de los algoritmos de detección de anomalías que emplean Deep Learning.
	\item Estudiar la teoría estadística que rodea el Machine Learning y el Deep Learning.
	\item Entender los fundamentos teóricos y el funcionamiento de los modelos implementados.
	\item Desarrollar una implementación de los modelos.
	\item Obtener una comparativa entre los modelos clásicos y los Deep Learning.
\end{itemize}

Todos estos objetivos han sido alcanzados en el desarrollo de este estudio, obteniendo conclusiones de calidad y líneas de trabajo futuro que se aplicarán a la continuidad del proyecto.